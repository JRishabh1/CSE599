% CVPR 2022 Paper Template
% based on the CVPR template provided by Ming-Ming Cheng (https://github.com/MCG-NKU/CVPR_Template)
% modified and extended by Stefan Roth (stefan.roth@NOSPAMtu-darmstadt.de)
\documentclass[10pt,twocolumn,letterpaper]{article}

%%%%%%%%% PAPER TYPE  - PLEASE UPDATE FOR FINAL VERSION
%\usepackage[review]{cvpr}      % To produce the REVIEW version
\usepackage{cvpr}              % To produce the CAMERA-READY version
%\usepackage[pagenumbers]{cvpr} % To force page numbers, e.g. for an arXiv version

% Include other packages here, before hyperref.
\usepackage{graphicx}
\usepackage{amsmath}
\usepackage{amssymb}
\usepackage{booktabs}


% It is strongly recommended to use hyperref, especially for the review version.
% hyperref with option pagebackref eases the reviewers' job.
% Please disable hyperref *only* if you encounter grave issues, e.g. with the
% file validation for the camera-ready version.
%
% If you comment hyperref and then uncomment it, you should delete
% ReviewTempalte.aux before re-running LaTeX.
% (Or just hit 'q' on the first LaTeX run, let it finish, and you
%  should be clear).
\usepackage[pagebackref,breaklinks,colorlinks]{hyperref}


% Support for easy cross-referencing
\usepackage[capitalize]{cleveref}
\crefname{section}{Sec.}{Secs.}
\Crefname{section}{Section}{Sections}
\Crefname{table}{Table}{Tables}
\crefname{table}{Tab.}{Tabs.}


%%%%%%%%% PAPER ID  - PLEASE UPDATE
\def\cvprPaperID{*****} % *** Enter the CVPR Paper ID here
\def\confName{CVPR}
\def\confYear{2022}


\begin{document}

%%%%%%%%% TITLE - PLEASE UPDATE
\title{Reinforcement Learning on Pokemon Emerald's Battle Tower}

\author{Rishabh Jain, Samarth Venkatesh \\
University of Washington \\
%Institution1 address\\
%{\tt\small firstauthor@i1.org}
% For a paper whose authors are all at the same institution,
% omit the following lines up until the closing ``}''.
% Additional authors and addresses can be added with ``\and'',
% just like the second author.
% To save space, use either the email address or home page, not both
% \and
% Samarth Venkatesh\\
% Institution2\\
% First line of institution2 address\\
% {\tt\small secondauthor@i2.org}
}
\maketitle

%%%%%%%%% ABSTRACT
%\begin{abstract}
    %lorem ipsum
%\end{abstract}

%%%%%%%%% BODY TEXT
\section{Introduction}
\label{sec:intro}

This paper examines implementing reinforcement learning to single-player challenges in Pokemon Emerald.
Reinforcement learning has previously been successful at learning how to accomplish tasks such as
walking \cite{Haarnoja} and even playing games \cite{Mnih}.

Previous attempts to use machine learning (ML) techniques on Pokemon have focused primarily on either on the PvP aspect \cite{Hu} or
playing through a game in its entirety \cite{Whiddy}. We attempt to explore a specific part
of single-player Pokemon that fuses these two aspects together; the Battle Tower. In the Battle Tower,
the player attempts to keep winning consecutive battles with their ``streak'' ending with a loss.
Since the only actions available in this challenge have to do with battling, we could potentially
gain insight into how a Q-Learning model learns and optimizes a subset of Pokemon.
Any revelations could be adapted and applied to Pokemon in a larger context or even other similar games.

Due to the smaller scope of this application as compared to player-vs-player or playing through the whole game,
we hope to achieve consistent results in winning a few battles without large training time,
which may reflect a potential in using reinforcement learning to maximize a streak. In particular, we hope to be able to get a streak (defined as $7$ consecutive victories in the Battle Tower with the same team). We later hope to train our model to win the Silver Symbol ($5$ consecutive streaks) and Gold Symbol ($10$ consecutive streaks) as these are tough challenges for a regular trainer.

%-------------------------------------------------------------------------
\section{Related Work}

\par{Reinforcement learning (RL), specifically Q-Learning, is a popular tool in training ML models to succeed in environments where information is incompletely known. The first major instance that demonstrated Q-Learning's for training agents to exceed human experts at a game with fixed rules was Mnih \etal's paper "Playing atari with deep reinforcement learning"~\cite{Mnih}.}
\par{The paper demonstrates that the principle challenges with RL algorithms is the timestep disparity between a given action and its resultant reward and the "sequences of highly correlated states" that the algorithm encounters rather than the independent data samples inherent to other models. The solution to this, the paper found, was to use a convolutional neural network to learn successful control policies and then smooth out the training distribution by sampling previous transitions.}
\par{Within Pokemon specifically, Sihao Hu \etal \cite{Hu} show demonstrate the viability of such an approach, using LLMs to autonomously play Pokemon and use in-context RL to refine policy when making decisions. Pablo Barros \etal \cite{Barros} use a 3 layer neural network model to learn to represent a battle and is trained using composed loss based on contrastive optimization to train a model to play effectively against specific opponents.}
\par{Finally, there are a few hobbyist explorations of designing ML models to play Pokemon battles \cite{Whiddy} and complete the single player game \cite{Compton}, which use various ML models including RL but due to the dynamic nature of a Pokemon battle and the many orders of magnitudes of possible game states, it is difficult to train an effective model. In addition, these models are effectively trained from scratch and as a result `waste" resources in re-learning policies that may have been investigated before. In addition, due to these models not being trained directly on the environment itself, they may not find the optimal policy for battling within the confines of the Battle Tower.}
%-------------------------------------------------------------------------
\section{Technical Challenges}

A primary hurdle for us to overcome before training can commence is the connection of our model to
the game itself. This will involve using OpenAi's Gym Retro and a GameBoy Advance Emulator to interact
with the game using Python. From there, we can begin creating and using classes for the game
to represent states, agents, etc. and begin training the model. 

%-------------------------------------------------------------------------
\section{Experiments}

Our model will center around maximizing a streak with a set team of Pokemon.
This removes the added complexity of team-building, training, and catching for
the model.

%-------------------------------------------------------------------------

%-------------------------------------------------------------------------


%-------------------------------------------------------------------------


%-------------------------------------------------------------------------



%-------------------------------------------------------------------------


%------------------------------------------------------------------------

%%%%%%%%% REFERENCES
{\small
\bibliographystyle{ieee_fullname}
\bibliography{egbib}
}

\end{document}
