% CVPR 2022 Paper Template
% based on the CVPR template provided by Ming-Ming Cheng (https://github.com/MCG-NKU/CVPR_Template)
% modified and extended by Stefan Roth (stefan.roth@NOSPAMtu-darmstadt.de)
\documentclass[10pt,twocolumn,letterpaper]{article}

%%%%%%%%% PAPER TYPE  - PLEASE UPDATE FOR FINAL VERSION
%\usepackage[review]{cvpr}      % To produce the REVIEW version
\usepackage{cvpr}              % To produce the CAMERA-READY version
%\usepackage[pagenumbers]{cvpr} % To force page numbers, e.g. for an arXiv version

% Include other packages here, before hyperref.
\usepackage{graphicx}
\usepackage{amsmath}
\usepackage{amssymb}
\usepackage{booktabs}


% It is strongly recommended to use hyperref, especially for the review version.
% hyperref with option pagebackref eases the reviewers' job.
% Please disable hyperref *only* if you encounter grave issues, e.g. with the
% file validation for the camera-ready version.
%
% If you comment hyperref and then uncomment it, you should delete
% ReviewTempalte.aux before re-running LaTeX.
% (Or just hit 'q' on the first LaTeX run, let it finish, and you
%  should be clear).
\usepackage[pagebackref,breaklinks,colorlinks]{hyperref}


% Support for easy cross-referencing
\usepackage[capitalize]{cleveref}
\crefname{section}{Sec.}{Secs.}
\Crefname{section}{Section}{Sections}
\Crefname{table}{Table}{Tables}
\crefname{table}{Tab.}{Tabs.}


%%%%%%%%% PAPER ID  - PLEASE UPDATE
\def\cvprPaperID{*****} % *** Enter the CVPR Paper ID here
\def\confName{CVPR}
\def\confYear{2022}


\begin{document}

%%%%%%%%% TITLE - PLEASE UPDATE
\title{Deep Learning with Little Data: Predicting Basketball Scores}

\author{Rishabh Jain, Samarth Venkatesh \\
University of Washington \\
%Institution1 address\\
%{\tt\small firstauthor@i1.org}
% For a paper whose authors are all at the same institution,
% omit the following lines up until the closing ``}''.
% Additional authors and addresses can be added with ``\and'',
% just like the second author.
% To save space, use either the email address or home page, not both
% \and
% Samarth Venkatesh\\
% Institution2\\
% First line of institution2 address\\
% {\tt\small secondauthor@i2.org}
}
\maketitle

%%%%%%%%% ABSTRACT
%\begin{abstract}
    %lorem ipsum
%\end{abstract}

%%%%%%%%% BODY TEXT

%-------------------------------------------------------------------------
\section{Abstract}

Fill in later

%-------------------------------------------------------------------------
\section{Introduction}
\label{sec:intro}

Deep Learning often requires large amounts of data and training. As technology as improved,
we have gained better systems such as GPT (ADD MORE) and others, in part due to the
feasibility of large scale training becoming available. However, some tasks are inherently
constrained by a lack of data. In this paper, we explore the applicability of Deep Learning
to one such task: predicting the outcome of basketball games. Specifically, we aim to
accurately predict the overall score of a match given the two teams playing against each other.

Due to the nature of basketball, teams are subject to change every year and performance
can be volatile. There is not guarantee that a team which performs well now will perform
just as well 5 years into the future (CITE EXAMPLE). Although data data regarding teams
may exist from many years back (CITE), it is hardly a good idea to use ancient statistics
when training a model for today's use; in other words, technological advancements
with regards to scaling won't allow for improvements in similar tasks constrained
in data naturally.

We aim to explore the possibility of using certain Deep Learning techniques (EXPAND)
to predict basketball game outcomes despite the lack of available data. Finding the right
features and hyperparameters will be of utmost importance since scalability is impossible.
We hope the results will be useful in applying deep learning to other tasks with limited
data availability (EXAMPLES?) and provide an insight into how ``rought draft'' models
can be created before scaling up.

%-------------------------------------------------------------------------
\section{Related Work}


%-------------------------------------------------------------------------
\section{Technical Challenges}


%-------------------------------------------------------------------------
\section{Experiments}


%-------------------------------------------------------------------------

%-------------------------------------------------------------------------


%-------------------------------------------------------------------------


%-------------------------------------------------------------------------



%-------------------------------------------------------------------------


%------------------------------------------------------------------------

%%%%%%%%% REFERENCES
{\small
\bibliographystyle{ieee_fullname}
\bibliography{references}
}

\end{document}
